%\part*{Lektion 4: Listen-Abstraktion, Generatoren und Ähnliches}

\section{Listen-Abstraktion/List-Comprehension}
\begin{itemize}
	\item Einfache Methode, um Listen zu erzeugen
	\begin{itemize}
		\item aus Strings, Dictionaries, Mengen, Bytes, ...
		\item bestehende Listen abändern
		\item bestehende Listen filtern
	\end{itemize}
	\item Alles auf einer Zeile
	\begin{itemize}
		\item übersichtlicher Code
	\end{itemize}
\end{itemize}

\subsection{Neue Liste aus einer bestehenden Liste ableiten}

Beispiel 1:\\
\begin{minipage}[t]{0.49\textwidth}
	\textbf{konventionell:}
	\lstinputlisting{listings/v4_list1.py}
\end{minipage}
\begin{minipage}[t]{0.02\textwidth} $ \quad $\end{minipage}
\begin{minipage}[t]{0.49\textwidth}
	\textbf{mit Listen-Abstraktion:}
	\lstinputlisting{listings/v4_list2.py}
\end{minipage}

Beispiel 2:\\
\begin{minipage}[t]{0.49\textwidth}
	\textbf{konventionell:}
	\lstinputlisting{listings/v4_list3.py}
\end{minipage}
\begin{minipage}[t]{0.02\textwidth} $ \quad $\end{minipage}
\begin{minipage}[t]{0.49\textwidth}
	\textbf{mit Listen-Abstraktion:}
	\lstinputlisting{listings/v4_list4.py}
\end{minipage}

\subsection{Bestehende Liste filtern}
Beispiel: Nur Früchte behalten, deren Name mit A, B oder C beginnen.
\lstinputlisting{listings/v4_list5.py}

\subsection{Liste von Zahlen => formatierter String}

\textbf{konventionell:}
\lstinputlisting{listings/v4_list6.py}
\textbf{mit Listen-Abstraktion:}
\lstinputlisting{listings/v4_list7.py}

\subsection{Liste der Schachbrettfelder}
\begin{minipage}[t]{0.39\textwidth}
	\textbf{konventionell:}
	\lstinputlisting{listings/v4_list8.py}
\end{minipage}
\begin{minipage}[t]{0.02\textwidth} $ \quad $\end{minipage}
\begin{minipage}[t]{0.59\textwidth}
	\textbf{mit Listen-Abstraktion:}
	\lstinputlisting{listings/v4_list9.py}
\end{minipage}


\subsection{Mengen-Abstraktion/Set Comprehension}

\subsubsection{Produkte zweier Zahlen}
\begin{minipage}[t]{0.39\textwidth}
	\textbf{konventionell:}
	\lstinputlisting{listings/v4_list10.py}
\end{minipage}
\begin{minipage}[t]{0.02\textwidth} $ \quad $\end{minipage}
\begin{minipage}[t]{0.59\textwidth}
	\textbf{mit Mengen-Abstraktion:}
	\lstinputlisting{listings/v4_list11.py}
\end{minipage}



\section{Iteratoren und Generatoren}
\begin{itemize}
	\item Iterator
	\begin{itemize}
		\item greift nacheinander auf die Elemente einer Menge von Objekten zu
		\item fundamentaler Bestandteil von Python, z.B. in \texttt{for}-Schleifen
	\end{itemize}
	\item Generator
	\begin{itemize}
		\item ist eine besondere Art, um einen Iterator zu implementieren
		\item wird mittels einer speziellen Funktion erzeugt
	\end{itemize}
\end{itemize}

\subsection{Iteratoren}
Iteratoren werden benutzt, um über einen Container zu iterieren.\\
Die for-SChleife erzeugt aus dem Listen-Objekt einen Iterator:
\lstinputlisting{listings/v4_iter1.py}
Das Container-Objekt muss die \texttt{\_\_iter\_\_()}-Funktion implementieren:
\lstinputlisting{listings/v4_iter2.py}
Iterator aus Liste erzeugen:
\lstinputlisting{listings/v4_iter3.py}
Ein Iterator muss auch die \texttt{\_\_next\_\_()}-Funktion implementieren:
\lstinputlisting{listings/v4_iter4.py}
Das nächste Element kann mit \texttt{next()} extrahiert werden:
\lstinputlisting{listings/v4_iter5.py}
... bis kein Element drin ist => \texttt{StopIteration}-Exception
\lstinputlisting{listings/v4_iter6.py}

\subsection{Generatoren}
Ein Generator ist auch ein Iterator.\\
Ein Generator wird erstellt, indem man eine Funktion aufruft, die eine oder mehrere \texttt{yield}-Answeisungen hat:
\lstinputlisting{listings/v4_iter7.py}
Bei der \texttt{yield}-Anweisung wird die Funktion (wie mit return) verlassen, aber Python merkt sich
\begin{itemize}
	\item den Zustand der lokalen Variable
	\item und wo der Generator verlassen wurde.
\end{itemize}
\lstinputlisting{listings/v4_iter8.py}

\subsubsection{Generator-Expression}
Ein Generator kann auch mit einem Ausdruck definiert werden:
\lstinputlisting{listings/v4_iter9.py}

\subsubsection{\texttt{send()}-Methode, Generator als Coroutine}
Die \texttt{send()}-Methode verhält sich im Prinzip wie die \texttt{next()}-Methode, aber sendet gleichzeitig noch einen Wert an den Generator:
\lstinputlisting{listings/v4_iter10.py}

\section{Listen und Tupel im Detail}
\begin{itemize}
	\item Tupel
	\begin{itemize}
		\item Packing
		\item Unpacking
	\end{itemize}
	\item Listen
	\begin{itemize}
		\item Elemente hinzufügen
		\item Sortieren
	\end{itemize}
\end{itemize}

\subsection{Tupel}
Leeres Tupel:
\lstinputlisting{listings/v4_tupel1.py}
Tupel mit einem Element:
\lstinputlisting{listings/v4_tupel2.py}
Mehrfachzuweisung:
\lstinputlisting{listings/v4_tupel3.py}
\begin{multicols}{2}
	Packing:
	\lstinputlisting{listings/v4_tupel4.py}
	\columnbreak
	Unpacking:
	\lstinputlisting{listings/v4_tupel5.py}
	\vfill\null
\end{multicols}
Packing mit Rest:
\lstinputlisting{listings/v4_tupel6.py}

\subsection{Listen}

\subsubsection{Element hinzufügen}
\lstinputlisting{listings/v4_tupel7.py}

\subsubsection{Mehrere Elemente hinzufügen}
\lstinputlisting{listings/v4_tupel8.py}
Mehrere Elemente zwischendrin einfügen:
\lstinputlisting{listings/v4_tupel9.py}

\subsubsection{Elemente ersetzen}
\lstinputlisting{listings/v4_tupel10.py}
Einzelnes Element:
\lstinputlisting{listings/v4_tupel11.py}
Einen ganzen Bereich:
\lstinputlisting{listings/v4_tupel12.py}

\subsubsection{Element entfernen}
\lstinputlisting{listings/v4_tupel13.py}

\subsection{Sortieren}
\texttt{sorted()} liefert eine neue sortierte Liste zurück:
\lstinputlisting{listings/v4_tupel14.py}
\texttt{sort()} modifiziert die Liste selbst (In-Place-Sortierung):
\lstinputlisting{listings/v4_tupel15.py}

\subsubsection{Umgekehrte Reihenfolge}
\lstinputlisting{listings/v4_tupel16.py}

\subsubsection{Mit spezieller Funktion}
\lstinputlisting{listings/v4_tupel17.py}

\subsubsection{\texttt{collections.deque}}
Falls ein Stack oder FIFO-Buffer mit folgenden Eigenschaften benötigt wird:
\begin{itemize}
	\item Thread-sicher
	\item Speicher-optimiert
	\item schnell
\end{itemize}
\url{https://docs.python.org/3/library/collections.html#collections.deque}
\lstinputlisting{listings/v4_tupel18.py}

\section{lambda, map, filter und reduce}
\begin{itemize}
	\item lambda
	\begin{itemize}
		\item anonyme Funktionen bauen
	\end{itemize}
	\item map, filter und reduce
	\begin{itemize}
		\item Hilfsmittel für die funktionale Programmierung
		\item auch mit List Comprehension möglich
	\end{itemize}
\end{itemize}

\subsection{lambda}
Mit \texttt{lambda} können anonyme Funktionen definiert werden.
\lstinputlisting{listings/v4_tupel19.py}

\subsection{map}
sequenz = map(funktion, sequenz)\\
Wendet die Funktion auf alle Elemente der Sequenz an und gibt die Resultate als Sequenz zurück.
\lstinputlisting{listings/v4_tupel20.py}
Funktion mit zwei Parametern benötigt zwei Listen:
\lstinputlisting{listings/v4_tupel21.py}

\subsection{filter}
sequenz = filter(funktion, sequenz)\\
Wendet die Funktion auf alle Elemente der Sequenz an und gibt nur diejenige Elemente zurück, für die die Funktion \texttt{True} liefert.
\lstinputlisting{listings/v4_tupel22.py}

\subsection{reduce}
resultat = reduce(funktion, sequenz)\\
Wendet die Funktion (mit zwei Parametern) fortlaufen auf die Sequenz an und liefert einen einzelnen Wert zurück.
\lstinputlisting{listings/v4_tupel23.py}