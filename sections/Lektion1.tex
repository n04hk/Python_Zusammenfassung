\part*{Lektion 1: Variablen und Datentypen}
\section{Datentypen}
\begin{itemize}
	\item Variablen bezeichnen keinen bestimmten Typ.
	\item Dynamische Typdeklaration
	\begin{itemize}
		\item \textbf{Automatische Zuweisung} des Datentyps bei Deklaration
		\item Datentyp ist während dem Programmablauf \textbf{veränderbar}
		\item Wert- und Typänderung erlaubt!
	\end{itemize}
\end{itemize}
\begin{table}[H]
\begin{threeparttable}
	\caption{Datentypen}
	\label{tab:T_Datentypen}
	\begin{tabular}{|l|l|l|}
		\hline 
		\textbf{Datentyp} &\textbf{Beschreibung} &\textbf{False-Wert}\\
		\hline
		\texttt{NoneType} &Indikator für nichts, keinen Wert &\texttt{None}\\ 
		\hline
		\textbf{Numerische Datentypen}&&\\
		\texttt{int} &Ganze Zahlen &\texttt{0}\\ 
		\texttt{float} &Gleitkommazahlen &\texttt{0.0}\\ 
		\texttt{bool} &Boolesche Werte &\texttt{False}\\ 
		\texttt{complex} &Komplexe Zahlen &\texttt{0 + 0j}\\ 
		\hline 
		\textbf{Sequenzielle Datentypen}&&\\
		\texttt{str} &Zeichenketten oder Strings &\texttt{''}\\
		\texttt{list} &Listen (veränderlich) &\texttt{[]}\\
		\texttt{tuple} &Tupel (unveränderlich) &\texttt{()}\\
		\texttt{bytes} &Sequenz von Bytes (unveränderlich) &\texttt{b''}\\
		\texttt{bytearray} &Sequenz von Bytes (veränderlich) &\texttt{bytearray(b'')}\\
		\hline
		\textbf{Assoziative Datentypen}&&\\
		\texttt{dict} &Dictionary (Schlüssel-Wert-Paare) &\texttt{\{\}}\\
		\hline
		\textbf{Mengen}&&\\
		\texttt{set} &Menge mit einmalig vorkommenden Objekten &\texttt{set()}\\
		\texttt{frozenset} &Wie set jedoch unveränderlich &\texttt{frozenset()}\\
		\hline
	\end{tabular}
\end{threeparttable}
\end{table}
\begin{itemize}
	\item Python erkennt den Datentyp automatisch
	\item Python ordnet jeder Variablen den Datentyp zu
	\item Datentypen prüfen:
	\begin{itemize}
		\item[\-] \texttt{type(object)}
		\item[\-] \texttt{isinstance(object, ct)}
	\end{itemize}
	\item Python achtet auf Typverletzungen
	\item Python kennt keine implizite Typumwandlung
\end{itemize}

\subsection[Numerische Datentypen]{Numerische Datentypen \tiny{Kap. 4}}
\begin{itemize}
	\item \texttt{bool}
	\item \texttt{int}
	\item \texttt{float}
	\item \texttt{complex}
\end{itemize}

\subsubsection{Arithmetische Operationen}
\begin{table}[H]
\begin{threeparttable}
	\caption{Arithmetische Operationen}
	\begin{tabular}{|c|l|}
		\hline 
		\textbf{Operator} &\textbf{Beschreibung}\\ 
		\texttt{x + y} &Summe von \texttt{x} und \texttt{y}\\ 
		\texttt{x - y} &Differenz von \texttt{x} und \texttt{y}\\ 
		\texttt{x * y} &Produkt von \texttt{x} und \texttt{y}\\ 
		\texttt{x / y} &Quotient von \texttt{x} und \texttt{y}\\  
		\texttt{x // y} &Ganzzahliger Quotient$^1$ von \texttt{x} und \texttt{y}\\ 
		\texttt{x \% y} &Rest der Division$^1$ von \texttt{x} durch \texttt{y}\\ 
		\texttt{+x} &Positives Vorzeichen\\ 
		\texttt{-x} &Negatives Vorzeichen\\ 
		\texttt{abs(x)} &Betrag von \texttt{x}\\ 
		\texttt{x**y} &Potenzieren, \texttt{x\textsuperscript{y}}\\ 
		\hline 
	\end{tabular}
\textsuperscript{1}Nicht definiert für den Datentyp \texttt{complex}
\end{threeparttable}
\end{table}

\begin{achtung}
	\texttt{x++} und \texttt{x-{}-} gibt es \textbf{nicht}, aber \texttt{x += 1, x -= 1, x *= 2, ...}
\end{achtung}
\begin{minipage}[t]{0.5\textwidth}
	\subsubsection{Vergleichende Operatoren}
	\begin{table}[H]
	\begin{threeparttable}
	\caption{Vergleichende Operatoren}
	\begin{tabular}{|c|l|}
		\hline 
		\textbf{Operator} &\textbf{Beschreibung}\\ 
		\hline 
		\texttt{==} &wahr, wenn \texttt{x} und \texttt{y} gleich sind\\ 
		\texttt{!=} &wahr, wenn \texttt{x} und \texttt{y} verschieden sind\\ 
		\texttt{<} &wahr, wenn \texttt{x} kleiner als \texttt{y} ist\textsuperscript{2}\\ 
		\texttt{<=} &wahr, wenn \texttt{x} kleiner oder gleich \texttt{y} ist\textsuperscript{2}\\ 
		\texttt{>} &wahr, wenn \texttt{x} grösser als \texttt{y} ist\textsuperscript{2}\\ 
		\texttt{>=} &wahr, wenn \texttt{x} grösser oder gleich \texttt{y} ist\textsuperscript{2}\\ 
		\hline 
	\end{tabular}
	\textsuperscript{2}Nicht definiert für den Datentyp \texttt{complex}\\
	\end{threeparttable}
	\end{table}
\end{minipage}
\begin{minipage}[t]{0.5\textwidth}
	\subsubsection{Bitweise Operatoren für den Datentypen \texttt{int}}
	\begin{table}[H]
	\begin{threeparttable}
	\caption{Bitweise Operatoren}
	\begin{tabular}{|c|l|}
		\hline 
		\textbf{Operator} &\textbf{Beschreibung}\\ 
		\hline 
		\texttt{x \& y} &bitweises UND von \texttt{x} und \texttt{y}\\ 
		\texttt{x | y} &bitweises ODER von \texttt{x} und \texttt{y}\\
		\texttt{x \textasciicircum y} &bitweises EXOR von \texttt{x} und \texttt{y}\\ 
		\texttt{\textasciitilde x} &bitweises Komplement von \texttt{x}\\ 
		\texttt{x << n} &Bit-Verschiebung um \texttt{n} Stellen nach links\\ 
		\texttt{x >> n} &Bit-Verschiebung um \texttt{n} Stellen nach rechts\\ 
		\hline 
	\end{tabular}
	\end{threeparttable}
	\end{table}
\end{minipage}

\subsubsection{Methoden nur dür den Datentyp \texttt{complex}}
\begin{table}[H]
\begin{threeparttable}
\caption{Methoden für complex}
\begin{tabular}{|l|l|}
	\hline 
	\textbf{Methode} &\textbf{Beschreibung}\\ 
	\hline 
	\texttt{x.real} &Realteil von \texttt{x} als Gleitkommazahl\\
	\texttt{x.imag} &Imaginärteil von \texttt{x} als Gleitkommazahl\\ 
	\texttt{x.conjugate()} &Liefert die zu \texttt{x} konjugiert komplexe Zahl\\  
	\hline 
\end{tabular}
\end{threeparttable}
\end{table}

\subsection[Sequentielle Datentypen]{Sequentielle Datentypen \tiny{Kap. 5}}
\begin{itemize}
	\item \texttt{str}
	\item \texttt{list}
	\item \texttt{tuple}
	\item \texttt{bytes}
	\item \texttt{bytearray}
\end{itemize}
\begin{table}[H]
\begin{threeparttable}
\caption{Methoden für sequenzielle Datentypen}
\begin{tabular}{|l|l|}
	\hline 
	\textbf{Operator} &\textbf{Beschreibung}\\ 
	\hline 
	\texttt{x in s} &Prüft, ob \texttt{x} in \texttt{s} enthalten ist.\\ 
	\texttt{x not in s} &Prüft, ob \texttt{x} nicht in \texttt{s} enthalten ist.\\
	\texttt{s + t} &Verkettung der beiden Sequenzen \texttt{s} und \texttt{t}.\\ 
	\texttt{s * n} &Verkettung von \texttt{n} Kopien der Sequenz \texttt{s}.\\ 
	\texttt{s[i]} &Liefert das \texttt{i}-te Element von s.\\ 
	\texttt{s[i:j]} &Liefert den Ausschnitt aus \texttt{s} von \texttt{i} bis \texttt{j}.\\
	\texttt{s[i:j:k]} &Liefert jedes \texttt{k}-te Element im Ausschnitt von \texttt{s} zwischen \texttt{i} und \texttt{j}.\\
	\texttt{len(s)} &Liefert die Anzahl Elemente in der Sequenz \texttt{s}.\\
	\texttt{max(s)} &Liefert das grösste Element in \texttt{s} (sofern eine Ordnung definiert ist).\\
	\texttt{min(s)} &Liefert das kleinste Element in \texttt{s} (sofern eine Ordnung definiert ist).\\
	\texttt{s.index(x)} &Liefert den Index des ersten Vorkommens von \texttt{x} in \texttt{s}.\\
	\texttt{s.count(x)} &Zählt, wie oft \texttt{x} in \texttt{s} vorkommt.\\
	\hline 
\end{tabular}
\end{threeparttable}
\end{table}

\subsection[Assoziative Datentypen]{Assoziative Datentypen \tiny{Kap. 6}}
\begin{itemize}
	\item dict
\end{itemize}
\begin{table}[H]
\begin{threeparttable}
\caption{Methoden für Assoziative Datentypen}
\begin{tabular}{|l|l|}
	\hline 
	\textbf{Operator} &\textbf{Beschreibung}\\ 
	\hline 
	\texttt{len(d)} &Liefert die Anzahl Schlüssel-Wert-Paare in \texttt{d}\\
	\texttt{d[k]} &Zugriff auf den Wert mit dem Schlüssel \texttt{k}\\
	\texttt{k in d} &Liefert \texttt{True}, wenn der Schlüssel \texttt{k} in \texttt{d} ist.\\
	\texttt{k not in d} &Liefert \texttt{True}, wenn der Schlüssel \texttt{k} nicht in \texttt{d} ist.\\
	\texttt{d.clear()} &Löscht alle Elemente aus dem Dictionary.\\
	\texttt{d.copy()} &Erstellt eine Kopie des Dictionaries.\\
	\texttt{d.get([k,[x]])} &Gibt den Wert des Schlüssels \texttt{k} zurück, ansonsten den Wert [x].\\
	\texttt{d.items()} &Gibt eine Liste der Schlüssel-Wert-Paare als Tuple zurück.\\
	\texttt{d.keys()} &Gibt eine Liste aller Schlüsselwerte zurück.\\
	\texttt{d.update(d2)} &Fügt ein Dictionary \texttt{d2} zu \texttt{d} hinzu.\\
	\texttt{d.pop(k)} &Entfernt das Element mit Schlüssel \texttt{k}.\\
	\texttt{d.popitem()} &Entfernt das zuletzt eingefügte Schlüssel-Wert-Paar.\\
	\texttt{d.setdefault(k,[x])} &Setzt den Wert [x] für den Schlüssel \texttt{k}.\\
	\hline 
\end{tabular}
\end{threeparttable}
\end{table}

\subsection[Mengen]{Mengen \tiny{Kap. 7}}
\begin{itemize}
	\item set
	\item frozenset
\end{itemize}

Ein set enthält eine ungeordnete Sammlung von einmaligen und unveränderlichen Elementen. In anderen Worten: Ein Element kann in einem set-Objekt nicht mehrmals vorkommen, was bei Listen und Tupel jedoch möglich ist.

\begin{table}[H]
\begin{threeparttable}
\caption{Methoden für Mengen}
\begin{tabular}{|l|l|}
	\hline 
	\textbf{Operator} &\textbf{Beschreibung}\\ 
	\hline 
	\texttt{s.add(el)} &Fügt ein neues unveränderliches Element (el) ein\\
	\texttt{s.clear()} &Löscht alle Elemente einer Menge.\\
	\texttt{s.copy()} &Erstellt eine Kopie der Menge.\\
	\texttt{s.difference(y)} &Die Menge s wird von y subtrahiert und in einer neuen Menge gespeichert.\\
	\texttt{s.difference\_update(y)} &Gleich wie s.difference(y) nur wird hier das Ergebnis direkt in s gespeichert.\\
	\texttt{s.discard(el)} &Das Element el wird aus der Menge s entfernt.\\
	\texttt{s.remove(el)} &Gleich wie s.discard(el) nur gibt es hier einen Fehler falls el nicht in s.\\
	\texttt{s.intersection(y)} &Liefert die Schnittmenge s und y.\\
	\texttt{s.isdisjoint(y)} &Liefert True falls Schnittmenge von s und y leer ist.\\
	\texttt{s.pop()} &Liefert ein beliebiges Element welches zugleich aus der Menge entfernt wird\\
	\hline 
\end{tabular}
\end{threeparttable}
\end{table}
