\part*{Lektion 2: Verzweigungen, Schleifen und Funktionen}
\section{Verzweigungen}
\subsection{\texttt{if}}
\lstinputlisting{listings/v2_if1.py}
Anweisungen 1 \& 2 nur ausführen, wenn die Bedingung \textbf{wahr} ist.\\
\begin{achtung}
	Alle Anweisungen im gleichen Codeblock müssen gleich eingerückt sein, z.B. mit vier Leerzeichen, sonst wird ein Fehler ausgegeben.<
\end{achtung}

\subsubsection{\texttt{if}-Anweisung mit \texttt{else}-Zweig}
\lstinputlisting{listings/v2_if2.py}
\begin{itemize}
	\item Anweisungen 1 \& 2, falls Bedingung \textbf{wahr}
	\item Anweisungen 3 \& 4, falls Bedingung \textbf{unwahr}
\end{itemize}
Für jeden Datentyp gibt es einen Wert, der als \textbf{unwahr} gilt:\\
\begin{tabular}{|l|l|}
	\hline
	\textbf{Datentyp} &\textbf{False-Wert}\\
	\hline
	\texttt{NoneType} &\texttt{None}\\
	\texttt{int} &\texttt{0}\\
	\texttt{float} &\texttt{0.0}\\
	\texttt{bool} &\texttt{False}\\
	\texttt{complex} &\texttt{0 + 0j}\\
	\texttt{str} &\texttt{'' oder \dq \dq  (leerer String)} \\
	\texttt{list} &\texttt{[]}\\
	\texttt{tuple} &\texttt{()}\\
	\texttt{bytes} &\texttt{b''}\\
	\texttt{bytearray} &\texttt{bytearray(b'')}\\
	\texttt{dict} &\texttt{\{\}}\\ 
	\texttt{set} &\texttt{set()}\\
	\texttt{frozenset} &\texttt{frozenset()}\\
	\hline
\end{tabular}

\subsubsection{\texttt{elif}-Zweige}
\lstinputlisting{listings/v2_if3.py}
\texttt{elif} = \texttt{else if}\\

\begin{achtung}
	Python kennt keine \texttt{switch}-\texttt{case}-Anweisung.
\end{achtung}

\section[Schleifen]{Schleifen \tiny{Kap. 10}}
\subsection{\texttt{while}}
\lstinputlisting{listings/v2_while1.py}
\begin{itemize}
	\item Anweisung1 wird wiederholt, solange die Bedingung \textbf{wahr} ist
	\item Einrücken des Codeblocks
\end{itemize}

\subsubsection{Durchlauf beenden und zurück nach oben}
\begin{achtung}
	Python kennt keine \texttt{do}-\texttt{while}-Schleife.
\end{achtung}

\lstinputlisting{listings/v2_while2.py}
\texttt{continue} beendet den aktuellen Durchlauf und springt nach oben.

\subsubsection{\texttt{while}-Schleife abbrechen}
\lstinputlisting{listings/v2_while3.py}
\texttt{break} bricht die \texttt{while}-Schleife vorzeitig ab

\subsubsection{\texttt{else}-Teil}
\lstinputlisting{listings/v2_while4.py}
\texttt{else}-Teil: wenn die Schleife \textbf{nicht} durch \texttt{break} abgebrochen wurde

\subsection{\texttt{for}}
\lstinputlisting{listings/v2_for1.py}
\begin{itemize}
	\item dient zur Iteration einer Sequenz
	\item Sequenz muss ein iterierbares Objekt sein:\\
	\texttt{list}, \texttt{tuple}, \texttt{dict}, \texttt{str}, \texttt{bytes}, \texttt{bytearray}, \texttt{set}, \texttt{frozenset}
\end{itemize}

\subsubsection{\texttt{else}-Teil}
\lstinputlisting{listings/v2_for2.py}
\texttt{else}-Teil wie bei der \texttt{while}-Schleife

\section[Funktionen]{Funktionen \tiny{Kap. 14}}
Python besitzt eine grosse Standard-Bibliothek, z.B.:
\lstinputlisting{listings/v2_func1.py}
\url{https://docs.python.org/3/library/}\\

und eingebaute Datentypen:\\
\url{https://docs.python.org/3/library/stdtypes.html}\\

und eingebaute Funktionen:\\
\url{https://docs.python.org/3/library/functions.html}

\subsection{Funktionsdefinition}
einfache Funktionsdefinition:\\
\lstinputlisting{listings/v2_func2.py}

Beispiel:\\
\lstinputlisting{listings/v2_func3.py}

\begin{itemize}
	\item Der Funktionsname kann frei gewählt werden
	\item Parameternamen durch Kommas trennen
	\item Codeblock gleichmässig einrücken
\end{itemize}

Der Rückgabewert der Funktion ist \texttt{None}, falls nichts angegeben wird.\\
\lstinputlisting{listings/v2_func4.py}

\texttt{return}-Anweisung beendet den Funktionsaufruf mit Rückgabewert:\\
\lstinputlisting{listings/v2_func5.py}
\begin{itemize}
	\item leere \texttt{return}-Anweisung liefert \texttt{None} zurück
	\item mehrere \texttt{return}-Anweisungen sind erlaubt, wie in C/C++
\end{itemize}

\subsection{Aufruf}
\lstinputlisting{listings/v2_func6.py}

\subsection{Weiteres}

\subsubsection{Standardwert für Parameter}
\lstinputlisting{listings/v2_func7.py}

\subsubsection{Mehrere Rückgabewerte}
\lstinputlisting{listings/v2_func8.py}

\subsubsection{Variable Anzahl von Argumenten}
\lstinputlisting{listings/v2_func9.py}

\subsubsection{Argumente entpacken}
\lstinputlisting{listings/v2_func10.py}

\subsubsection{Beliebige Schlüsselwort-Parameter}
\lstinputlisting{listings/v2_func11.py}

\subsubsection{Schlüsselwortparameter entpacken}
\lstinputlisting{listings/v2_func12.py}

\subsubsection{Globale Variablen}
\lstinputlisting{listings/v2_func13.py}

\subsubsection{Docstring - Funktion dokumentieren}
PEP 257 - Docstring Conventions \url{https://www.python.org/dev/peps/pep-0257}
\lstinputlisting{listings/v2_func14.py}

\subsubsection{Call-by-object-reference}
mit veränderlichen Objekten:
\lstinputlisting{listings/v2_func15.py}
mit unveränderlichen Objekten:
\lstinputlisting{listings/v2_func16.py}
