%\part*{Lektion 2: Verzweigungen, Schleifen und Funktionen}
\section[Verzweigungen]{Verzweigungen \tiny{Kap. 9}}
\subsection{\texttt{if}}
\lstinputlisting{listings/v2_if1.py}
%Anweisungen 1 \& 2 nur ausführen, wenn die Bedingung \textbf{wahr} ist.\\
\begin{achtung}
	Alle Anweisungen im gleichen Codeblock müssen gleich eingerückt sein, z.B. mit vier Leerzeichen, sonst wird ein Fehler ausgegeben.
\end{achtung}

\begin{minipage}[t]{0.59\textwidth}
	\subsubsection{\texttt{if}-Anweisung mit \texttt{else}-Zweig}
	\lstinputlisting{listings/v2_if2.py}
\end{minipage} 
\begin{minipage}[t]{0.02\textwidth} $ \quad $\end{minipage}
\begin{minipage}[t]{0.39\textwidth}
	
	\subsubsection{\texttt{elif}-Zweige}
	\lstinputlisting{listings/v2_if3.py}
\end{minipage}
$ \quad $\\[8pt]
Für jeden Datentyp gibt es einen Wert, der als \textbf{unwahr} gilt. Siehe \autoref{tab:T_Datentypen} auf der Seite \pageref{tab:T_Datentypen}.

\begin{achtung}
	Python kennt keine \texttt{switch}-\texttt{case}-Anweisung.
\end{achtung}

\section[Schleifen]{Schleifen \tiny{Kap. 10}}
\subsection{\texttt{while}}
\lstinputlisting{listings/v2_while1.py}

\begin{achtung}
	Python kennt keine \texttt{do}-\texttt{while}-Schleife.
\end{achtung}

\begin{minipage}[t]{0.49\textwidth}
\subsubsection{\texttt{continue}}
\lstinputlisting{listings/v2_while2.py}
%\texttt{continue} beendet den aktuellen Durchlauf und springt nach oben.
\end{minipage}
\begin{minipage}[t]{0.02\textwidth} $ \quad $\end{minipage}
\begin{minipage}[t]{0.49\textwidth}
\subsubsection{\texttt{break}}
\lstinputlisting{listings/v2_while3.py}
%\texttt{break} bricht die \texttt{while}-Schleife vorzeitig ab
\end{minipage}

\begin{minipage}[t]{0.44\textwidth}
	\subsubsection{\texttt{else}-Teil}
	\lstinputlisting{listings/v2_while4.py}
\end{minipage}
\begin{minipage}[t]{0.02\textwidth} $ \quad $\end{minipage}
\begin{minipage}[t]{0.54\textwidth}
	\subsection{\texttt{for}}
	\lstinputlisting{listings/v2_for1.py}
	\begin{itemize}
		\item dient zur Iteration einer Sequenz
		\item Sequenz muss ein iterierbares Objekt sein:\\
		\texttt{list}, \texttt{tuple}, \texttt{dict}, \texttt{str}, \texttt{bytes}, \texttt{bytearray}, \texttt{set}, \texttt{frozenset}
		\item Die \texttt{for}-Schleife kennt auch \texttt{continue} und \texttt{break} somit gibt es auch einen \texttt{else} teil analog zur \texttt{while}-schleife.
	\end{itemize}
\end{minipage}





\section[Funktionen]{Funktionen \tiny{Kap. 14}}
\lstinputlisting{listings/v2_func1.py}
Standart Bibliotheken:  \url{https://docs.python.org/3/library/}

und eingebaute Datentypen:  \url{https://docs.python.org/3/library/stdtypes.html}

und eingebaute Funktionen:  \url{https://docs.python.org/3/library/functions.html}

\subsection{Funktionsdefinition}
einfache Funktionsdefinition:\\
\lstinputlisting{listings/v2_func2.py}

\begin{minipage}[t]{0.49\textwidth}
	Beispiel:
	\lstinputlisting{listings/v2_func3.py}
	\lstinputlisting{listings/v2_func5.py}
\end{minipage}
\begin{minipage}[t]{0.02\textwidth} $ \quad $\end{minipage}
\begin{minipage}[t]{0.49\textwidth}
	\begin{itemize}
		\item Der Funktionsname kann frei gewählt werden
		\item Parameternamen durch Kommas trennen
		\item Codeblock gleichmässig einrücken
		\item Der Rückgabewert der Funktion ist \texttt{None}, falls nichts angegeben wird.
		\item \texttt{return}-Anweisung beendet den Funktionsaufruf
		\item es sind mehrere \texttt{return}-Anweisungen sind erlaubt, wie in C/C++
	\end{itemize}
\end{minipage}

\subsection{Aufruf}
\lstinputlisting{listings/v2_func6.py}

\subsection{Weiteres}
\begin{minipage}[t]{0.49\textwidth}
	\subsubsection{Standardwert für Parameter}
	\lstinputlisting{listings/v2_func7.py}
\end{minipage}
\begin{minipage}[t]{0.02\textwidth} $ \quad $\end{minipage}
\begin{minipage}[t]{0.49\textwidth}
	\subsubsection{Mehrere Rückgabewerte}
	\lstinputlisting{listings/v2_func8.py}
\end{minipage}

\begin{minipage}[t]{0.49\textwidth}
	\subsubsection{Variable Anzahl von Argumenten}
	\lstinputlisting{listings/v2_func9.py}
\end{minipage}
\begin{minipage}[t]{0.02\textwidth} $ \quad $\end{minipage}
\begin{minipage}[t]{0.49\textwidth}
	\subsubsection{Argumente entpacken}
	\lstinputlisting{listings/v2_func10.py}
\end{minipage}

\begin{minipage}[t]{0.56\textwidth}
	\subsubsection{Beliebige Schlüsselwort-Parameter}
	\lstinputlisting{listings/v2_func11.py}
\end{minipage}
\begin{minipage}[t]{0.02\textwidth} $ \quad $\end{minipage}
\begin{minipage}[t]{0.42\textwidth}
	\subsubsection{Schlüsselwortparameter entpacken}
	\lstinputlisting{listings/v2_func12.py}
\end{minipage}

\begin{minipage}[t]{0.49\textwidth}
	\subsubsection{Globale Variablen}
	\lstinputlisting{listings/v2_func13.py}
\end{minipage}
\begin{minipage}[t]{0.02\textwidth} $ \quad $\end{minipage}
\begin{minipage}[t]{0.49\textwidth}
	\subsubsection{Docstring - Funktion dokumentieren}
	PEP 257 - Docstring Conventions\\
	\url{https://www.python.org/dev/peps/pep-0257}
	\lstinputlisting{listings/v2_func14.py}	
\end{minipage}

\subsubsection{Call-by-object-reference}
\begin{minipage}[t]{0.49\textwidth}
	mit veränderlichen Objekten:
	\lstinputlisting{listings/v2_func15.py}
\end{minipage}
\begin{minipage}[t]{0.02\textwidth} $ \quad $\end{minipage}
\begin{minipage}[t]{0.49\textwidth}
	mit unveränderlichen Objekten:
	\lstinputlisting{listings/v2_func16.py}
\end{minipage}



