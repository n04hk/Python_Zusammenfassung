%\part*{Lektion 3: Exceptions, Dateien und Strings}
\section[Exceptions]{Exceptions \tiny{Kap. 20}}
\begin{itemize}
	\item Fehler (\url{https://docs.python.org/3/tutorial/errors.html}) können auftreten, z.B.:
	\lstinputlisting{listings/v3_exception1.py}
	und führen zu einem Abbruch des Programms
	\item Fehler können abgefangen werden:
	\lstinputlisting{listings/v3_exception2.py}
\end{itemize}

\subsection{Unspezifische Exceptions abfangen}
Nicht empfohlen, da auch Exceptions geschluckt werden, die weitergegeben werden sollten, z.B. KeyboardInterrupt.\\
\lstinputlisting{listings/v3_exception3.py}

\subsection{Master Beispiel}
\lstinputlisting{listings/v3_exception9.py}
\begin{tabular}{lll}
	\textbf{Eingabe} & \textbf{File}  & \textbf{Ausgaben}\\
	5 		& existiert & Alles Okey $ \quad $ Auf wiedersehen $ \quad $ Prog. laeuft noch\\
	0 		& existiert & Eingabe darf nicht 0 sein! $ \quad $ Auf wiedersehen $ \quad $ Prog. laeuft noch\\
	200		& existiert & Error: Wert ist zu Gross! $ \quad $ Auf wiedersehen $ \quad $ Prog. laeuft noch\\
	'10.-' 	& existiert & Err: invalid literal for int() with base 10: '10.-' $ \quad $Auf wiedersehen $ \quad $ Prog. laeuft noch\\
	$[$5, 1$]$ 	& existiert & Auf wiedersehen $ \quad $ Lange Fehlermeldung\\ 
	5		& exist. nicht & Err: [Errno 2] No such file or directory: 'dat.txt' $ \quad $ Auf wiedersehen $ \quad $ Prog. laeuft noch
\end{tabular}\\[10pt]
'Auf wiedersehen' wird immer ausgegeben, 'Prog. laeuft noch' wird dann ausgegeben wenn kein Fehler auftrat oder dieser abgefangen wurde.

\pagebreak
\section[Dateien]{Dateien \tiny{Kap. 11}}

\subsection{Datei öffnen}
\begin{itemize}
	\item Datei mit der \texttt{open()}-Funktion öffnen:
	\lstinputlisting{listings/v3_datei1.py}
	\item Weitere Parameter findet man in der Hilfe (\url{https://docs.python.org/3/library/functions.html#open}):
	\lstinputlisting{listings/v3_datei2.py}
\end{itemize}

\subsection{Dateien lesen und schreiben}

\lstinputlisting{listings/v3_datei0.py}


\begin{minipage}[t]{0.49\textwidth}
	\subsubsection{\texttt{with}-Anweisung}
	Dateien sollten besser mit einer \texttt{with}-Anweisung geöffnet werden, dadurch wird sie am ende des Blocks automatisch geschlossen.
\end{minipage}
\begin{minipage}[t]{0.02\textwidth} $ \quad $\end{minipage}
\begin{minipage}[t]{0.49\textwidth}
	\lstinputlisting{listings/v3_datei7.py}
\end{minipage}



\subsubsection{glob}
\lstinputlisting{listings/v3_datei10.py}

\subsubsection{os.path}
\lstinputlisting{listings/v3_datei11.py}

\section{Strings}

\subsection[Stringformatierung]{Stringformatierung \tiny{Kap. 12}}
Stringformatierung benötigt man um Daten hübsch auszugeben oder systematisch abzuspeichern.\\
\begin{minipage}[t]{0.49\textwidth}
	\lstinputlisting{listings/v3_strings1.py}
\end{minipage}
\begin{minipage}[t]{0.02\textwidth} $ \quad $ \end{minipage}
\begin{minipage}[t]{0.49\textwidth}
	\lstinputlisting{listings/v3_strings2.py}
\end{minipage}

\subsubsection{im C-Stil (à la printf)}
\lstinputlisting{listings/v3_strings3.py}

\subsubsection{mit \texttt{format()}}
\lstinputlisting{listings/v3_strings23.py}

\begin{minipage}[t]{0.49\textwidth}
	\subsubsection{mit Stringliterale}
	\lstinputlisting{listings/v3_strings10.py}
\end{minipage}
\begin{minipage}[t]{0.02\textwidth} $ \quad $ \end{minipage}
\begin{minipage}[t]{0.49\textwidth}
	\subsubsection{mit \texttt{string}-Methoden}
	\lstinputlisting{listings/v3_strings11.py}	
\end{minipage}

\begin{minipage}[t]{0.39\textwidth}
	\subsection{Alles über Strings \tiny{Kap. 19}}
	\lstinputlisting{listings/v3_strings24.py}
	\subsubsection{Strings kombinieren}
	\lstinputlisting{listings/v3_strings17.py}
\end{minipage}
\begin{minipage}[t]{0.02\textwidth} $ \quad $\end{minipage}
\begin{minipage}[t]{0.59\textwidth}
	\subsubsection{Strings aufspalten}
	$\bullet$ \texttt{split()}
	\lstinputlisting{listings/v3_strings15.py}
	$\bullet$ \texttt{splitlines()}
	\lstinputlisting{listings/v3_strings16.py}
\end{minipage}


\begin{minipage}[t]{0.49\textwidth}
	\subsubsection{Suchen von Teilstrings}
	\lstinputlisting{listings/v3_strings18.py}
\end{minipage}
\begin{minipage}[t]{0.02\textwidth} $ \quad $\end{minipage}
\begin{minipage}[t]{0.49\textwidth}
	\subsubsection{Ersetzen von Teilstrings}
	\lstinputlisting{listings/v3_strings19.py}
\end{minipage}

\subsubsection{Strings bereinigen}
\lstinputlisting{listings/v3_strings20.py}

\begin{minipage}[t]{0.49\textwidth}
	\subsubsection{Klein- und Grossbuchstaben}
	\lstinputlisting{listings/v3_strings21.py}
\end{minipage}
\begin{minipage}[t]{0.02\textwidth} $ \quad $\end{minipage}
\begin{minipage}[t]{0.49\textwidth}
	\subsubsection{Strings testen}
	\lstinputlisting{listings/v3_strings22.py}
\end{minipage}

