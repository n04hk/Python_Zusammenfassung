\part*{Lektion 3: Exceptions, Dateien und Strings}
\section{Exceptions}
\begin{itemize}
	\item Fehler (\url{https://docs.python.org/3/tutorial/errors.html}) können auftreten, z.B.:
	\lstinputlisting{listings/v3_exception1.py}
	\item und führen zu einem Abbruch des Programms
	\item Fehler können abgefangen werden:
	\lstinputlisting{listings/v3_exception2.py}
\end{itemize}

\subsection{Unspezifische Exceptions abfangen}
Nicht empfohlen, da auch Exceptions geschluckt werden, die weitergegeben werden sollten, z.B. KeyboardInterrupt.\\
\lstinputlisting{listings/v3_exception3.py}

\subsection{Abfangen mehrerer Exceptions}
\lstinputlisting{listings/v3_exception4.py}
mehrfache Ausnahmen gruppieren:\\
\lstinputlisting{listings/v3_exception5.py}

\subsection{else-Teil}
\lstinputlisting{listings/v3_exception6.py}

\subsection{finally-Teil}
\lstinputlisting{listings/v3_exception7.py}

\subsection{Exceptions generieren}
\lstinputlisting{listings/v3_exception8.py}

\section{Dateien}

\subsection{Datei öffnen}
\begin{itemize}
	\item Datei mit der \texttt{open()}-Funktion öffnen:
	\lstinputlisting{listings/v3_datei1.py}
	\item Weitere Parameter findet man in der Hilfe (\url{https://docs.python.org/3/library/functions.html#open}):
	\lstinputlisting{listings/v3_datei2.py}
\end{itemize}

\subsection{Dateien lesen und schreiben}
\begin{itemize}
	\item Datei lesen:
	\lstinputlisting{listings/v3_datei3.py}
	\item Datei schreiben:
	\lstinputlisting{listings/v3_datei4.py}
	\item Datei schliessen:
	\lstinputlisting{listings/v3_datei5.py}
\end{itemize}

\subsubsection{Datei lesen}
\begin{itemize}
	\item mit \texttt{read()}
	\lstinputlisting{listings/v3_datei6.py}
	\item besser mit der \texttt{with}-Anweisung
	\lstinputlisting{listings/v3_datei7.py}
	\item Variante mit \texttt{readlines()}
	\lstinputlisting{listings/v3_datei8.py}
\end{itemize}

\subsubsection{Datei schreiben}
\lstinputlisting{listings/v3_datei9.py}

\subsubsection{glob}
\lstinputlisting{listings/v3_datei10.py}

\subsubsection{os.path}
\lstinputlisting{listings/v3_datei11.py}

\section{Strings}

\subsection{Stringformatierung}
\begin{itemize}
	\item Stringformatierung benötigt man um Daten hübsch auszugeben
	\lstinputlisting{listings/v3_strings1.py}
	\item oder systematisch abzuspeichern
	\lstinputlisting{listings/v3_strings2.py}
\end{itemize}

\subsubsection{im C-Stil (à la printf)}
\lstinputlisting{listings/v3_strings3.py}

\subsubsection{mit \texttt{format()}}
\begin{itemize}
	\lstinputlisting{listings/v3_strings4.py}
	\item mit Index:
	\lstinputlisting{listings/v3_strings5.py}
	\item mit Index und Format:
	\lstinputlisting{listings/v3_strings6.py}
	\item links-/rechtsbündig oder zentriert:
	\lstinputlisting{listings/v3_strings7.py}
	\item mit Schlüsselwortparameter:
	\lstinputlisting{listings/v3_strings8.py}
	\item mit Dictionary:
	\lstinputlisting{listings/v3_strings9.py}
\end{itemize}

\subsubsection{mit Stringliterale}
\lstinputlisting{listings/v3_strings10.py}

\subsubsection{mit \texttt{string}-Methoden}
\lstinputlisting{listings/v3_strings11.py}

\subsection{Alles über Strings}
\begin{itemize}
	\item Unicode-Nummer => Zeichen
	\lstinputlisting{listings/v3_strings12.py}
	\item Zeichen => Unicode-Nummer
	\lstinputlisting{listings/v3_strings13.py}
	\item String => bytes
	\lstinputlisting{listings/v3_strings14.py}
\end{itemize}

\subsubsection{Strings aufspalten}
\begin{itemize}
	\item \texttt{split()}
	\lstinputlisting{listings/v3_strings15.py}
	\item \texttt{splitlines()}
	\lstinputlisting{listings/v3_strings16.py}
\end{itemize}

\subsubsection{Strings kombinieren}
\lstinputlisting{listings/v3_strings17.py}

\subsubsection{Suchen von Teilstrings}
\lstinputlisting{listings/v3_strings18.py}

\subsubsection{Ersetzen von Teilstrings}
\lstinputlisting{listings/v3_strings19.py}

\subsubsection{Strings bereinigen}
\lstinputlisting{listings/v3_strings20.py}

\subsubsection{Klein- und Grossbuchstaben}
\lstinputlisting{listings/v3_strings21.py}

\subsubsection{Strings testen}
\lstinputlisting{listings/v3_strings22.py}