%\part*{Lektion 5: Reguläre Ausdrücke}

\section[Reguläre Ausdrücke]{Reguläre Ausdrücke \tiny{Kap. 36}}
\begin{itemize}
	\item Regular Expressions (RE, regex, regex pattern)
	\item Bilden eine kleine Programmiersprache innerhalb von Python
	\item Sind verfügbar im \texttt{re}-Modul (\url{https://docs.python.org/3/library/re.html}) - \texttt{import re}
	\item Definieren Muster, auf die nur gewisse Strings passen, z.B.:
	\begin{itemize}
		\item Entspricht die angegebene E-Mail-Adresse dem Muster?
		\item Welche Wörter im Text beginnen mit $"$ver-$"$ und enden mit $"$-en$"$?
	\end{itemize}
	\item Die meisten Buchstaben und Zeichen passen auf sich selbst:
	\begin{itemize}
		\item \texttt{test} passt genau auf sich selbst
	\end{itemize}
	\item Folgende Metazeichen haben eine spezielle Bedeutung:
	\begin{itemize}
		\item[\-] . $\textasciicircum$ $\textdollar$ * + ? \{ \} [ ] $\textbackslash$ | ()
		\item[\-] . passt auf alle Zeichen, ausser Newline-Zeichen
	\end{itemize}
\end{itemize}

\subsection{Zeichen-Klassen}
\begin{itemize}
	\item Die Metazeichen [ und ] definieren eine Zeichen-Klasse
	\begin{description}
		\item[abc] passt auf alle Zeichen a, b oder c
		\item[a-z] passt auf einen Kleinbuchstaben
		\item[a-zA-Z] passt auf einen Klein- oder Grossbuchstaben
	\end{description}
	\item Andere Metazeichen sind in Zeichen-Klasse nicht aktiv:
	\begin{description}
		\item[akm$\textdollar$] passt auf die Zeichen a, k, m oder \textdollar, wobei $\textdollar$ sonst ein Metazeichen ist. 
	\end{description}
	\item Das $\textasciicircum$-Zeichen definiert die komplementäre Menge:
	\begin{description}
		\item[$\textasciicircum$abc] passt auf alle Zeichen, ausser a, b und c
	\end{description}
	\item Vordefinierte Zeichen-Klassen:\\
	\begin{tabular}{lll}
		$\textbackslash$d& Dezimalziffer& [0-9]\\ 
		$\textbackslash$D& keine Dezimalziffer& [$\textasciicircum$0-9]\\ 
		$\textbackslash$s& Leer- oder Steuerzeichen& [$\textbackslash$t$\textbackslash$n$\textbackslash$r$\textbackslash$f$\textbackslash$v]\\ 
		$\textbackslash$S& kein Leer- oder Steuerzeichen& [$\textasciicircum$$\textbackslash$t$\textbackslash$n$\textbackslash$r$\textbackslash$f$\textbackslash$v]\\ 
		$\textbackslash$w& Unicode-Wortzeichen (auch Umlaute)& [a-zA-Z0-9\_]\\ 
		$\textbackslash$W& kein Wortzeichen& [$\textasciicircum$a-zA-Z0-9\_]\\ 
	\end{tabular}
	\item Verwendung in Zeichen-Klassen:\\
	\begin{tabular}{ll}
		[A-Fa-f$\textbackslash$d]& passt auf eine Hexadezimalziffer\\
		$[\textbackslash s,.]$& passt auf ein Leerzeichen, Komma oder Punkt\\
	\end{tabular}
\end{itemize}

\subsection{Wiederholungen (Quantoren)}
\begin{tabular}{lcl}
	0 oder mehr     & * 		& ca*t passt auf ct, cat, caat, ...\\
					&			& a[0-9]*b passt auf ab, a538b, a0b, ...\\
	1 oder mehr	    & + 		& ca+t passt nicht auf ct, aber cat, caat, ...\\
	0 oder 1-mal	& ? 		& 10k?m passt auf 10m oder 10km\\
	m bis n-mal		& \{m,n\}	& ab\{2,3\}c passt auf abbc oder abbbc\\
					& \{3\}		& $\rightarrow$ genau 3-mal\\
					& \{3,\}	& $\rightarrow$ mindestens 3-mal\\
\end{tabular}


\subsection{Übereinstimmungen}
Funktionen, die Übereinstimmungen liefern:\\
\begin{tabular}{ll}
	\texttt{re.match()}& Prüft, ob die RA am Stringanfang passt.\\
	& Return \texttt{None} oder \texttt{match}-Objekt.\\
	\texttt{re.search()}& Sucht erstes Auftreten vom RA im String.\\
	& Return \texttt{None} oder \texttt{match}-Objekt.\\
	\texttt{re.findall()}& Findet alle Teilstrings, die mit dem RA passen.\\
	& Return Liste mit allen Teilstrings.\\
	\texttt{re.finditer()}& Findet alle Teilstrings, die mit dem RA passen.\\
	& Return Iterator, welcher \texttt{match}-Objekte liefert.\\
\end{tabular}

	
\subsubsection{\texttt{match}-Objekt}
Memberfunktionen eines \texttt{match}-Objekts:\\
\begin{tabular}{ll}
	\texttt{group()}& Return Teilstring, welcher mit dem RA passt.\\
	\texttt{start()}& Return Startposition des Teilstrings.\\
	\texttt{end()}& Return Endpostion des Teilstrings.\\
	\texttt{span()}& Return Tupel mit (\texttt{start, end}).
\end{tabular}

\subsubsection{Übereinstimmungen finden}
\begin{minipage}[t]{0.49\textwidth}
	\texttt{re.match(pattern, string, flags=0)}
	\lstinputlisting{listings/v5_ra2.py}
	\texttt{re.search(pattern, string, flags=0)}
	\lstinputlisting{listings/v5_ra3.py}
	
\end{minipage}
\begin{minipage}[t]{0.02\textwidth} $ \quad $\end{minipage}
\begin{minipage}[t]{0.49\textwidth}
	\texttt{re.findall(pattern, string, flags=0)}
	\lstinputlisting{listings/v5_ra4.py}
	\texttt{re.finditer(pattern, string, flags=0)}
	\lstinputlisting{listings/v5_ra5.py}
\end{minipage}


\subsection{Modifizierungen}
Funktionen, die Modifizierungen durchführen:\\
\begin{tabular}{ll}
	\texttt{re.split()}& Trennt den String dort, wo der RA passt.\\
	& Gibt eine Liste mit den Teilstrings zurück.\\
	\texttt{re.sub()}& Ersetzt jeden Teilstring, der mit dem RA passt.\\
	& Gibt den neuen String zurück.\\
	\texttt{re.subn()}& Gleich wie bei \texttt{re.sub()},\\
	& gibt aber einen Tupel (Neuer String, Anzahl) zurück.\\
\end{tabular}\\[10pt]
\texttt{re.split(pattern, string, maxsplit=0, flags=0)}\\
Der String wird überall dort getrennt, wo ein Teilstring auf den RA passt, z.B.: zwischen den Wörtern.
\lstinputlisting{listings/v5_ra6.py}
\texttt{re.sub(pattern, repl, string, count=0, flags=0)}\\
Jeder Teilstring, der auf den RA passt, wird mit dem repl-String ersetzt:
\lstinputlisting{listings/v5_ra7.py}
Mit \texttt{count} kann die Anzahl Ersetzungen limitiert werden:
\lstinputlisting{listings/v5_ra8.py}

\pagebreak[1]
Eine Funktion bei \texttt{repl} angeben. Das Argument ist ein \texttt{match}-Objekt, der Rückgabewert muss ein String sein.
\lstinputlisting{listings/v5_ra9.py}
\texttt{re.subn(pattern, repl, string, count=0, flags=0)}\\
Gleich wie bei \texttt{re.sub()}, aber es wird ein Tupel mit dem neuen String und die Anzahl der Ersetzungen zurückgegeben:
\lstinputlisting{listings/v5_ra10.py}

\subsection{Gruppierung}
\begin{itemize}
	\item Teile eines Ausdrucks können gruppiert werden
	\item Normale Gruppierung mit \texttt{()}\\
	\begin{tabular}{ll}
		(ab)+c& passt auf abc, ababc, ...\\
		(ab)\textbackslash1& mit Rückwärtsreferenz, passt auf abab\\
	\end{tabular}
	\item Benannte Gruppierung mi (?P<...>)\\
	\begin{tabular}{ll}
		(?P<zahl>\textbackslash d+)& passt auf 13\\
		(?P<zahl>\textbackslash d+)-(?P=zahl)& mit Referenz, passt auf 13-13\\
	\end{tabular}
	\item Passive Gruppierung (non-capturing group) mit (?:...)\\
	\begin{tabular}{ll}
		(?:ab)& passt auf ab, Gruppe wird nicht hinterlegt\\
	\end{tabular}
\end{itemize}

\texttt{match-Objekt}\\
Mittels der \texttt{groups()}-Memberfunktion eines \texttt{match}-Objektes erhält man ein Tupel mit den Übereinstimmungen der einzelnen Gruppen.\\
Folgende Funktionen liefern ein \texttt{match}-Objekt: \texttt{re.match()}, \texttt{re.search()}, und \texttt{re.finditer()}.
\lstinputlisting{listings/v5_ra11.py}

\texttt{re.findall()}\\
Falls Gruppen im RA angegeben werden, dann werden nur die Übereinstimmungen der Gruppen als Liste von Tupeln zurückgegeben.
\lstinputlisting{listings/v5_ra13.py}
verschachtelte Gruppen, öffnende Klammern definieren die Reihenfolge
\lstinputlisting{listings/v5_ra14.py}
\texttt{re.split()}\\
Falls Gruppen im RA angegeben werden, dann werden auch die Übereinstimmungen der Gruppen in der Liste zurückgegeben.
\lstinputlisting{listings/v5_ra15.py}
\texttt{re.sub()}\\
\lstinputlisting{listings/v5_ra16.py}

\subsubsection{Weitere Metazeichen}
Spezielle Prüfzeichen (belegen keinen Platz):\\
\begin{tabular}{ll}
	$\textbar$& x|y passt entweder auf x oder y\\
	$\textasciicircum$& steht für den Anfang des Strings\\
	& oder für den Anfang jeder Zeile (bei flag=re.MULTILINE)\\
	$\textdollar$& steht für das Ende des Strings\\
	& oder für das Ende jeder Zeile (bei flag=re.MULTILINE)\\
	$\textbackslash$A& steht für den Anfang des Strings\\
	$\textbackslash$Z& steht für das Ende des Strings\\
	$\textbackslash$b& steht für eine Wortgrenze\\
	$\textbackslash$B& steht für das Gegenteil von $\textbackslash$b\\
\end{tabular}\\

\lstinputlisting{listings/v5_ra17.py}


\subsubsection{Look-around Assertions}
\begin{itemize}
	\item positive, vorausschauende Annahme\\
	\begin{tabular}{ll}
		(?=Ausdruck)& Ausdruck muss hier folgen\\
	\end{tabular}
	\item negative, vorausschauende Annahme\\
	\begin{tabular}{ll}
		(?!Ausdruck)& Ausdruck darf hier nicht folgen\\
	\end{tabular}
	\item positive, nach hinten schauende Annahme\\
	\begin{tabular}{ll}
		(?<=Ausdruck)& Ausdruck muss hier vorangehen\\
	\end{tabular}
	\item negative, nach hinten schauende Annahme\\
	\begin{tabular}{ll}
		(?<=!Ausdruck)& Ausdruck darf hier nicht vorausgehen\\
	\end{tabular}
\end{itemize}

\lstinputlisting{listings/v5_ra21.py}
